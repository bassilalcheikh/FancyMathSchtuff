\documentclass[a4paper]{article}
\usepackage{amsmath}
\usepackage{amsthm}
\newtheorem{thm}{Theorem}
\usepackage[english]{babel}
\usepackage[utf8]{inputenc}
\usepackage{graphicx}
\usepackage[colorinlistoftodos]{todonotes}
\usepackage{amsfonts}

\theoremstyle{definition}
\newtheorem{defn}{Definition}[section]
\newtheorem{conj}{Conjecture}[section]
\newtheorem{exmp}{Example}[section]

\title{Moment Generating Functions}

\author{Bassil Alcheikh}

\date{\today}
\begin{document}
\maketitle

%\begin{abstract}
%These notes will define the special statistical functions known as %\textit{moment generating functions} and prove several related theorems.
%\end{abstract}

\section{Introduction}
The motivation behind moment generating functions is to find (or prove) a particular probability distribution $p(y)$ for some random variable $Y$. Thus,
\begin{itemize}
   \item  for $p(y), \: m(t)$ exists $\implies m(t)$ is unique
   \item  moment generating functions (MFGs) can be used to establish equivalence between two probability distributions
\end{itemize}
\section{Definitions}

\begin{defn} % Definition 2.1
The \textbf{$k^{th}$ moment of a random variable $Y$} taken about the origin is equal to E$(Y^k)$ and is denoted by $\mu'_{k}.$
\end{defn}
\bigbreak
\begin{defn} % Definition 2.2
The \textbf{$k^{th}$ moment of a random variable $Y$} taken about its mean is equal to E[$(Y-\mu)^k$] and is denoted by $\mu'_{k}.$
\end{defn}
\bigbreak
\begin{defn} % Definition 2.3
The \textbf{moment-generating function $m(t)$ for a random variable $Y$}: $m(t)=$ E$(e^{tY})$; $m(t)$ exists if $\exists b \in \mathbb{R}^+: m(t)$ is finite for $|t| \le b.$
\bigbreak
Consider the following:
$$e^{tY} = 1+ty+\frac{(ty)^2}{2!}+\frac{(ty)^3}{3!}+ \cdots$$
\quad Assuming $\mu'_k$ is finite for $k = 1,2,3,...$
\begin{align*}
E(e^{tY})&= \sum_y e^{tY}p(y) \\
{}&= \sum_y \bigg[1+ty+\frac{(ty)^2}{2!}+\frac{(ty)^3}{3!}+ \cdots\bigg] \\
{}&= \sum_y p(y) +\sum_y ty\cdot p(y) + \sum_y\frac{(ty)^2}{2!}+\cdots \\
{}&= 1+ t\mu'_1 + \frac{t^2}{2!}\mu'_2 + \frac{t^3}{3!}\mu'_3 + \cdots
\end{align*}
\end{defn} % definition 2.3 finally over...
\bigbreak
% Here comes the theorem...
\begin{thm} If $m(t)$ exists, then $\forall k \in \mathbb{Z}^+,$
$$\frac{\delta^k m(t)}{\delta t^k} \bigg]_{t=0} = m^{(k)}(0) = \mu_k'.$$
\end{thm}
% Proof:
\begin{proof} Begin with the definition of $m(t)$, and then take its $k^{th}$ derivative:
\begin{align*}
\\m(t)&=E(e^{tY})
\\{}&=1+ t\mu'_1 + \frac{t^2}{2!}\mu'_2 + \frac{t^3}{3!}\mu'_3 + \cdots
\\m'(t)&=\mu'_1+ t\mu'_2 + \frac{t^2}{2!}\mu'_3 + \frac{t^3}{3!}\mu'_4+\cdots
\\m''(t)&=\mu'_2+ t\mu'_3 + \frac{t^2}{2!}\mu'_4 + \frac{t^3}{3!}\mu'_5+\cdots
\\\text{In general terms,}
\\m^{k}(t)&=\mu'_{(k)}+ t\mu'_{(k+1)} + \frac{t^2}{2!}\mu'_{(k+2)} + \frac{t^3}{3!}\mu'_{(k+3)}+\cdots
\\\text{Evaluated at }t=0,
\\m^{k}(0)&=\mu'_{(k)}+ 0\cdot \mu'_{(k+1)} + \frac{0^2}{2!}\mu'_{(k+2)} + \frac{0^3}{3!}\mu'_{(k+3)}+\cdots
\\m^{k}(0)&=\mu'_{(k)}
\end{align*}
\end{proof}

\section{Examples}

\end{document}
